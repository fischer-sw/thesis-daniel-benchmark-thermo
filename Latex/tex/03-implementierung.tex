\documentclass[../thesis.tex]{subfiles}



\begin{document}
	
% set hydrogen bonding charakteristics %from https://tex.stackexchange.com/questions/577565/how-to-draw-a-dotted-hydrogen-bond-with-chemfig
\makeatletter % from: https://tex.stackexchange.com/a/101263/134144
\tikzset{
	dot diameter/.store in=\dot@diameter,
	dot diameter=1pt,
	dot spacing/.store in=\dot@spacing,
	dot spacing=3.0pt,
	dots/.style={
		line width=\dot@diameter,
		line cap=round,
		dash pattern=on 0pt off \dot@spacing
	}
}
\makeatother

\chapter{Implementierung}
\label{chp: implementierung}

Die Implementierung lässt sich in drei verschiedene Bestandteile aufteilen, welche in den folgenden Abschnitten näher erläutert werden.

\section{Datenbank}

Die Klasse Datenbank dient zur Bereitstellung und Speicherung aller benötigten Daten. Diese Daten beinhalten sowohl die der Experimente, als auch die des Modells. Die Klasse Datenbank erzeugt im gleichnamigen Ordner für jedes Gemisch eine eigene Datei, in der alle Daten im \textit{.json} Format abgelegt sind. Der beispielhafte Aufbau einer Gemischdatei ist in \autoref{} dargestellt.

Der Schlüssel "BAC" dient der Klassifizierung des Gemisches, wie in \autoref{sec: gemischklassifikation} beschrieben. Das Tabellenblatt auf dem die Daten in der Originaldatei zu finden sind als Wert des Schlüssels "sheet" gespeichert. Analog zum Aufbau der Originaldatei werden unter den weiteren Schlüsseln die Daten der gemessenen Größen gespeichert. Alle Messreihen werden als Liste unter dem Punkt "[Variablenname]" gespeichert.

Eine Messreihe ist durch ihre Quelle unter dem Punkt "reference", die Parameter der Messung unter dem Punkt "params", und die Messdaten unter dem Punkt "measurements" charakterisiert.

Somit stehen alle Daten in computerlesbaren Format bereit.

\section{Modell}

\section{Bewertung}

\end{document}
