\documentclass[../thesis.tex]{subfiles}
\begin{document}

\chapter{Einleitung}
\label{chp:Motivation}
Thermodynamische Modelle und insbesondere Zustandsgleichungen haben in den vergangen Jahren und Jahrzehnten stark an Bedeutung gewonnen. Sie dienen der Bestimmung von Flüssigkeitseigenschaften wie beispielsweise der Dichte oder der Zusammensetzung. Durch die kontinuierliche Weiterentwicklungen und Verbesserungen sind inzwischen auch energetische Größen, wie z.B. Enthalpien und Wärmekapazitäten berechenbar. Diese Berechnungen sind sowohl für Reinstoffe als auch für Gemische durchführbar.
Zustandsgleichungen bilden das Herzstück einer jeden Software zum Design und Optimierung von verfahrenstechnischen Anwendungen und Prozessen. Durch das Lösen von Massen- und Energiebilanzen können Prozessgrößen bestimmt und verschieden Verfahrensansätze miteinander verglichen werden. Eine möglichst realistische Beschreibung der Stoffeigenschaften ist dafür besonders wichtig. Um dies zu gewährleisten ist eine Validierung der verwendeten Modelle mit verschiedenen experimentell ermittelten Stoffdaten notwendig. Am dies über einen großen Anwendungsbereich aus verschiedenen Drücken und Temperaturen zu ermöglichen ist eine einfach zu verwendende Zusammenstellung von experimentellen Stoffdaten notwendig.
Basierend auf dieser Forderung wurden von einer Forschergruppe experimentelle Daten zu verschieden Stoffen gesammelt. Diese Stoffe sind in verschiedene Stoffklassen unterteilt worden. Zusätzlich zu der Sammlung der Daten wurde ein Bewertungssystem für ein zu testendes thermodynamisches Modell entwickelt, sodass Modelle miteinander verglichen werden können. \cite{jaubert2020benchmark}

In dieser Arbeit soll es um die Implementation der entwickelten Methode und deren Anwendung mit der Stoffdatensoftware \textit{TREND} gehen.

\end{document}
