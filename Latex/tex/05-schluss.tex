\documentclass[../thesis.tex]{subfiles}

\begin{document}

\chapter{Zusammenfassung und Ausblick}

In dieser Arbeit wurde die in \cite{jaubert2020benchmark} vorgestellte Methode zur Bewertung thermodynamischer Modelle in der Programmiersprache \texttt{python} implementiert. Zur Berechnung der Modelldaten wurde die Stoffdatensoftware \texttt{TREND} mit dem bereits vorhandenen \texttt{python-Interface} benutzt. Als Beispiel und um die Implementation testen zu können, wurde die bereits in \texttt{TREND} vorhandene Zustandsgleichung \texttt{PSRK} verwendet.

Phasengleichgewichtsdaten, sowie energetische Zustandsgrößen spielen bei der Auslegung und Optimierung von verfahrenstechnischen Anlagen und Prozessen eine wichtige Rolle. Die korrekte Beschreibung dieser Größen ist somit für einen großen Temperatur- und Druckbereich für möglichst viele verschiedene Gemische von hohem Interesse. Die implementierte Methode bietet die Möglichkeit neu entwickelte Modelle umfangreich zu testen, um vorhandene Fehler bei der Berechnung dieser Zustandsgrößen zu identifizieren. Nur ausreichend getestete Modelle können das Vertrauen eines potentiellen Anwenders gewinnen, sodass die implementierte Methode, einen wertvollen Beitrag dazu leisten kann, die Zeit von der Entwicklung bis zur Anwendung eines thermodynamischen Modells zu verkürzen.

Auf kurze oder lange Sicht scheint es sinnvoll, die hier implementierte Methode direkt in \texttt{TREND} zu integrieren. Das entwickelte Tool bietet auf dem Weg dorthin die Möglichkeit Modelle bereits bewerten und vergleichen zu können.

\end{document}
